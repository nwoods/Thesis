%!TEX root = ../nwoods_thesis.tex

\chapter{Analysis Strategy}

\section{Background Estimation}
Reducible backgrounds for four-lepton events typically have two prompt leptons and two other objects---typically jet fragments, sometimes photons---which are misidentified as prompt leptons.
The largest source of background contamination is from evens in which a {\PZ} boson is produced in association with a photon and a jet, a leptonically-decaying $\PW$ boson and a jet, or two jets.
There is also a contribution from {\TTbar} events in which both top quarks decay to a lepton, a neutrino, and a {\Pqb}~quark jet.
For simplicity, the two sets of processes are not treated separately in what follows, and are collectively labeled ``$\PZ+\PX$'' events.

The contributions of the reducible backgrounds to the selected four-lepton signal samples are evaluated using the tight-to-loose ``fake rates'' method, described more fully in Ref.~\cite{Chatrchyan:2013mxa}.
In this procedure, the likelihood of a nonprompt (``fake'') object to be misidentified as a prompt lepton is estimated and applied to control regions enriched with $\PZ+\PX$ events to estimate their contribution to the signal region.
The lepton misidentification rate $f_\ell\left(\pt^\ell, \eta^\ell\right)$ is measured from a sample of $\PZ + \ell_\text{fake}$ events, where the {\PZ} candidate is selected as in the signal region but with $\left| m_{\ell\ell} - m_\PZ \right| < 10\GeV$, and the $\ell_\text{fake}$ object is a lepton candidate that passes relaxed ID requirements as defined in Section~\ref{sec:looseID}, with no isolation or tight ID requirements applied.
Events with three prompt leptons can contaminate this control region, because the {non-\PZ} lepton is assumed fake.
To avoid the resulting bias in the misidentification rate, the contribution of $\WZ \to 3\ell\nu$ to the $\PZ+\ell_\text{fake}$ sample is estimated from a simulated sample and subtracted.

The misidentification rate is defined as the fraction of $\ell_\text{fake}$ candidates which pass full lepton identification and isolation critera, in bins of $\pt$ and $\eta$.
One should note that this is not a probability in the usual sense, and there is not a simple physical interpretation of these misidentification rates.
Figure~\ref{fig:fakerates} shows the misidentification rates for electrons and muons separately as a function of {\pt} and $\eta$.

\begin{figure}[htbp]
  \begin{center}
    \includegraphics[width=0.35\textwidth]{methods/eFakeRate_pt.png}
    \includegraphics[width=0.35\textwidth]{methods/eFakeRate_eta.png} \\
    \includegraphics[width=0.35\textwidth]{methods/mFakeRate_pt.png}
    \includegraphics[width=0.35\textwidth]{methods/mFakeRate_eta.png}
    \caption[Misidentification rates for electrons and muons]{
      Fake rate for electrons~(top) and muons~(bottom) as a function of $\pt$~(left) and $\eta$~(right).
      }\label{fig:fakerates}
  \end{center}
\end{figure}

To estimate the total reducible background yield, the misidentification rates are applied to two $\PZ+\PX$ enriched control samples, each containing a {\PZ}~boson candidate passing all signal region requirements plus two more lepton candidates which pass the relaxed identification criteria and would make a second passing {\PZ} boson candidate except that one or both fail the full identification or isolation criteria.
The sample with one failing lepton, called the ``3P1F'' sample for ``3 prompt 1 fake,'' covers the contribution from {\WZ} events, while the sample with both leptons in the second {\PZ} boson failing (``2P2F'') covers $\PZ+\text{jets}$ and {\TTbar} events.
Both have the contribution from {\ZZ} events in which one or two prompt leptons fail identification or isolation criteria removed based on simulated samples.
The fake object transfer factor
\begin{equation}
  F_\ell\left(\pt^\ell,\eta^\ell\right) = \frac{f_\ell\left(\pt^\ell, \eta^\ell\right)}{1 - f_\ell\left(\pt^\ell, \eta^\ell\right)}
\end{equation}
is the ratio of nonprompt objects passing the relaxed and full selection criteria, and thus serves as an extrapolation factor between control sample yields and signal sample yields.

The total reducible background yield is thus
\begin{equation}\label{eq:bkgYield}
  N_\text{bkg} = \sum_{\ell \in \text{3P1F}} F_\ell\left(\pt^\ell,\eta^\ell\right) - \sum_{\ell_1,\ell_2 \in 2P2F} F_{\ell_1}\left(\pt^{\ell_1},\eta^{\ell_1}\right)F_{\ell_2}\left(\pt^{\ell_2},\eta^{\ell_2}\right),
\end{equation}
where subtraction of signal contamination is implicit.
The minus sign in Eq.~\ref{eq:bkgYield} prevents double-counting of $\PZ+2\text{jets}$ events in which one jet fragment is misidentified.

There are also irreducible background contributions from {\TTZ} and {\WWZ} events, which can have four prompt leptons.
Expected yields for these processes are taken from simulation.



\section{Systematic Uncertainties}
Who knows?



\section{Fiducial and Total Cross Section Calculation}

\subsection{Signal Strength Extraction}
Fitting


\subsection{\texorpdfstring{$\mathrm{Z} \to 4\ell$}{Z to 4l} Branching Fraction}
It is tiny



\section{Differential Cross Sections}

\subsection{Unfolding}
IT'S FREQUENTIST\@!


\subsection{Propagation of Systematic Uncertainties}
A small pain



\section{VBS Signal Extraction}
BDTs and other hip things



\section{Anomalous Gauge Coupling Searches}
