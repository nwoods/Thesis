%!TEX root = ../nwoods_thesis.tex

\chapter{The Standard Model}

\section{Introduction}
The standard model (SM) is a theory---or rather, set of several related theories---that encapsulates everything we currently know about matter and its interactions at a fundamental level.
This is a remarkable claim: in the particle physicist's reductionist worldview, subatomic particle interactions are the substrate underlying the rest of reality, with all other physics, and by extension everything else, arising as emergent properties.
Of course, the SM is a remarkable theory, making detailed predictions on a wide range of topics that have matched data in essentially every experiment over roughly four decades.
The small number of known phenomena outside the SM are topics on which it makes no prediction; it is fully self-consistent and on the topics it covers, it is consistent with data to the precision achievable by any experiment to date.
It is arguably the best-confirmed theory in the history of science despite making some of the boldest, broadest, and most precise predictions.
It is generally believed that future advances will add to it, explain its free parameters, or find some underlying structure, not contradict it.

The following sections give a general overview of the SM and related topics that serve as background material for the four-lepton processes described in more detail in the following chapters.
This will include discussions of the particle content of the SM and the gauge structure that leads to particle interactions, the spontaneous symmetry breaking mechanism that leads to the specific structure of the electroweak sector of the SM, diboson processes, and the SM's limitations and how they might be addressed.
Some details will also be given about the proton-proton interactions used to probe particle interactions at high energies.
More complete information may be found in a number of texts, including Refs.~\cite{Griffiths:111880,Halzen:1984mc,Peskin:1995ev}.
Detailed treatment of the mathematical and field theoretical underpinnings of particle physics, including the theory's aesthetically pleasing grounding in natural symmetries, is also discussed in a number of books, including Refs.~\cite{Peskin:1995ev,Srednicki:1019751,Donoghue:238727}.



\section{Matter and Force}
Does this need subsections? Not the way it's structured in my head right now, but maybe it could have one subsection for fermions and another for gauge bosons.



\section{Electroweak Symmetry Breaking and the Higgs Boson}
Everybody's favorite fundamental scalar boson



\section{Proton-Proton Collisions}\label{sec:pp}
Bang



\section{Diboson Physics}
Really only ZZ, but you get the point

\subsection{Vector Boson Scattering}
Scatter scatter


\section{Limitations and Possible Extensions}
It misses a few things

\subsection{Anomalous Gauge Couplings}
Pro tip: we won't see them



\section{Topics Covered In This Thesis}
