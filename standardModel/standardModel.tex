%!TEX root = ../nwoods_thesis.tex

\chapter{The Standard Model}

\section{Introduction}

The standard model (SM) is a theory---or rather, set of several related theories---that encapsulates everything we currently know about matter and its interactions at a fundamental level.
This is a remarkable claim: in the particle physicist's reductionist worldview, subatomic particle interactions are the substrate underlying the rest of reality, with all other physics, and by extension everything else, arising as emergent properties.
Of course, the SM is a remarkable theory, making detailed predictions on a wide range of topics that have matched data in essentially every experiment over roughly four decades.
The small number of known phenomena outside the SM are topics on which it makes no prediction; it is fully self-consistent and on the topics it covers, it is consistent with data to the precision achievable by any experiment to date.
It is arguably the best-confirmed theory in the history of science despite making some of the boldest, broadest, and most precise predictions.
It is generally believed that future advances will add to it, explain its free parameters, or find some underlying structure, not contradict it.

The following sections give a general overview of the SM and related topics that serve as background material for the four-lepton processes described in more detail in the following chapters.
This will include discussions of the particle content of the SM and the gauge structure that leads to particle interactions, the spontaneous symmetry breaking mechanism that leads to the specific structure of the electroweak sector of the SM, diboson processes, and the SM's limitations and how they might be addressed.
Some details will also be given about the proton-proton interactions used to probe particle interactions at high energies.
More complete information may be found in a number of texts, including Refs.~\cite{Griffiths:111880,Halzen:1984mc,Peskin:1995ev}.
Detailed treatment of the mathematical and field theoretical underpinnings of particle physics, including the theory's aesthetically pleasing grounding in natural symmetries, is also discussed in a number of books, including Refs.~\cite{Halzen:1984mc,Peskin:1995ev,Srednicki:1019751,Donoghue:238727}.
Unless otherwise stated, everything that follows uses units such that $c = \hbar = 1$, where $c$ is the speed of light and $\hbar$ is the reduced Planck's constant $\hbar = h / 2\pi$.



\section{Matter and Force}

In the SM, matter is made of fermions (particles with half-integer spin; in fact all SM fundamental fermions have spin $\frac{1}{2}$) which interact by exchanging gauge bosons (integer spin; spin 1 for the SM force carriers).
The fermions come in two broad categories, leptons and quarks.
All the quarks and half the lepton types carry electric charge and are therefore subject to interactions through the electromagnetic force, described by quantum electrodynamics (QED).
In a QED interaction, two charged particles exchange a photon, which carries the momentum transferred from one charged particle to the other.
The photon is a spin-1 gauge boson that is electrically neutral itself and massless, explaining why electromagnetic forces are long-range.
Because it is so simple, QED was the first theory of fundamental force to be worked out in detail, and it served as the template for the theories of the other forces.
The Feynman diagram at leading order (LO) in perturbation theory for the simplest QED interaction, the so-called Drell-Yan process, in which a fermion-antifermion pair ($\Pf\Paf$, where {\Pf} can be any charged fermion) annihilates and produces a different pair ($\Pf'\Paf'$) is shown in in Fig.~\ref{fig:drellYanDiagram}.
Our conventions for Feynman diagrams will be that time increases from left to right, fermions are straight lines with an arrow whose direction differentiates fermions (arrow points right) from antifermions (arrow points left), and photons are shown as wavy lines.

\begin{figure}[htbp]
  \vspace{1em}
  \begin{center}
    \begin{fmffile}{drellYanDiagram}
      \begin{fmfgraph*}(0.6,0.3) % chktex 36
        \fmfleft{i1,i2}
        \fmfright{o1,o2}
        \fmflabel{$\Pf$}{i1}
        \fmflabel{$\Paf$}{i2}
        \fmflabel{$\Paf'$}{o1}
        \fmflabel{$\Pf'$}{o2}
        \fmf{fermion}{i1,v1,i2}
        \fmf{fermion}{o1,v2,o2}
        \fmf{photon,label=$\Pa$}{v1,v2}
      \end{fmfgraph*}
    \end{fmffile}
    \vspace{1em}
    \caption[Feynman diagram of an electromagnetic Drell-Yan interaction]{
        Feynman diagram of fermion-antifermion scattering through an electromagnetic interaction, resulting in another fermion-antifermion pair.
        This is also known as a Drell-Yan process.
        At center-of-mass energies near and above the {\PZ} boson mass, {\PZ}-{\Pa} interference becomes nonnegligible.
      }\label{fig:drellYanDiagram}
  \end{center}
\end{figure}

There are six types of quarks which fall into three ``generations:'' up and down ({\Pqu} and {\Pqd}, first generation); charm and strange ({\Pqc} and {\Pqs}, second generation); and top and bottom ({\Pqt} and {\Pqb})\footnote{Top and bottom quarks are sometimes called truth and beauty by more romantic particle physicists.}.
Quark masses increase with each successive generation.
Up-type quarks ({\Pqu}, {\Pqc}, {\Pqt}) have electric charge $+2/3$ (in units of the positron charge) while down-type quarks have $-1/3$.
Quarks are the building blocks of hadron, including $\Pq\Paq'$ bound states called mesons and $\Pq\Pq'\Pq''/\Paq\Paq'\Paq''$ bound states called baryons, of which protons ($\Pqu\Pqu\Pqd$) and neutrons ($\Pqu\Pqd\Pqd$) are the most familiar.




\section{Electroweak Symmetry Breaking and the Higgs Boson}
Everybody's favorite fundamental scalar boson



\section{Proton-Proton Collisions}\label{sec:pp}
Bang



\section{Diboson Physics}
Really only ZZ, but you get the point

\subsection{Vector Boson Scattering}
Scatter scatter


\section{Limitations and Possible Extensions}
It misses a few things

\subsection{Anomalous Gauge Couplings}
Pro tip: we won't see them



\section{Topics Covered In This Thesis}
