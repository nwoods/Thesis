%!TEX root = ../nwoods_thesis.tex

\chapter{The Standard Model}

\section{Introduction}
The standard model (SM) is a fully self-consistent theory---or rather, set of several related theories---that encapsulates everything we currently know about matter and its interactions at a fundamental level.
This is a remarkable claim: in the particle physicist's reductionist worldview, subatomic particle interactions are the substrate underlying the rest of reality, with all other physics, and by extension everything else, arising as emergent properties.
Of course, the SM is a remarkable theory, making detailed predictions on a wide range of topics that have matched data in essentially every experiment over roughly four decades.
The small number of known phenomena outside the SM are topics on which it makes no prediction; it is fully self-consistent to the precision achievable by any experiment to date.
It is arguably the best-confirmed theory in the history of science.




\section{Matter and Force}
Does this need subsections? Not the way it's structured in my head right now, but maybe it could have one subsection for fermions and another for gauge bosons.



\section{Electroweak Symmetry Breaking and the Higgs Boson}
Everybody's favorite fundamental scalar boson



\section{Proton-Proton Collisions}\label{sec:pp}
Bang



\section{Diboson Physics}
Really only ZZ, but you get the point

\subsection{Vector Boson Scattering}
Scatter scatter


\section{Limitations and Possible Extensions}
It misses a few things

\subsection{Anomalous Gauge Couplings}
Pro tip: we won't see them



\section{Topics Covered In This Thesis}
