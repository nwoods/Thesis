%!TEX root = ../nwoods_thesis.tex

\chapter{The Standard Model}

\section{Introduction}

The standard model (SM) is a theory---or rather, set of several related theories---that encapsulates everything we currently know about matter and its interactions at a fundamental level.
This is a remarkable claim: in the particle physicist's reductionist worldview, subatomic particle interactions are the substrate underlying the rest of reality, with all other physics, and by extension everything else, arising as emergent properties.
Of course, the SM is a remarkable theory, making detailed predictions on a wide range of topics that have matched data in essentially every experiment over roughly four decades.
The small number of known phenomena outside the SM are topics on which it makes no prediction; it is fully self-consistent and on the topics it covers, it is consistent with data to the precision achievable by any experiment to date.
It is arguably the best-confirmed theory in the history of science despite making some of the boldest, broadest, and most precise predictions.
It is generally believed that future advances will add to it, explain its free parameters, or find some underlying structure, not contradict it.

The following sections give a general overview of the SM and related topics that serve as background material for the four-lepton processes described in more detail in the following chapters.
This will include discussions of the particle content of the SM and the gauge structure that leads to particle interactions, the spontaneous symmetry breaking mechanism that leads to the specific structure of the electroweak sector of the SM, diboson processes, and the SM's limitations and how they might be addressed.
Some details will also be given about the proton-proton interactions used to probe particle interactions at high energies.
More complete information may be found in a number of texts, including Refs.~\cite{Griffiths:111880,Halzen:1984mc,Peskin:1995ev}.
Detailed treatment of the mathematical and field theoretical underpinnings of particle physics, including the theory's aesthetically pleasing grounding in natural symmetries, is also discussed in a number of books, including Refs.~\cite{Halzen:1984mc,Peskin:1995ev,Srednicki:1019751,Donoghue:238727}.
Unless otherwise stated, everything that follows uses units such that $c = \hbar = 1$, where $c$ is the speed of light and $\hbar$ is the reduced Planck's constant $\hbar = h / 2\pi$.



\section{Matter and Force}

In the SM, matter is made of fermions (particles with half-integer spin; in fact all SM fundamental fermions have spin $\frac{1}{2}$) which interact by exchanging gauge bosons (integer spin; spin 1 for the SM force carriers).
Table~X lists the fundamental particles and some of their properties.
With the exception of the neutral bosons, all particles have a corresponding antiparticle which is the same except that all its quantum numbers have opposite sign.
The fermions come in two broad categories, leptons and quarks.
All the quarks and half the lepton types carry electric charge and are therefore subject to interactions through the electromagnetic force, described by quantum electrodynamics (QED).
In a QED interaction, two charged particles exchange a photon, which carries the momentum transferred from one charged particle to the other.
The photon is a spin-1 gauge boson that is electrically neutral itself and massless, explaining why electromagnetic forces are long-range.
Because it is so simple, QED was the first theory of fundamental force to be worked out in detail, and it served as the template for the theories of the other forces.
The Feynman diagram at leading order (LO) in perturbation theory for a simple QED interaction, the so-called Drell-Yan process, in which a fermion-antifermion pair ($\Pf\Paf$, where {\Pf} can be any charged fermion) annihilates and produces a different pair ($\Pf'\Paf'$) is shown in in Fig.~\ref{fig:drellYanDiagram}.
Our conventions for Feynman diagrams will be that time increases from left to right, fermions are straight lines with an arrow whose direction differentiates fermions (arrow points right) from antifermions (arrow points left), and photons are shown as wavy lines.

\begin{figure}[htbp]
  \vspace{1em}
  \begin{center}
    \begin{fmffile}{drellYanDiagram}
      \begin{fmfgraph*}(0.6,0.3) % chktex 36
        \fmfleft{i1,i2}
        \fmfright{o1,o2}
        \fmflabel{$\Pf$}{i1}
        \fmflabel{$\Paf$}{i2}
        \fmflabel{$\Paf'$}{o1}
        \fmflabel{$\Pf'$}{o2}
        \fmf{fermion}{i1,v1,i2}
        \fmf{fermion}{o1,v2,o2}
        \fmf{photon,label=$\Pa$}{v1,v2}
      \end{fmfgraph*}
    \end{fmffile}
    \vspace{1em}
    \caption[Feynman diagram of an electromagnetic Drell-Yan interaction]{
        Feynman diagram of fermion-antifermion scattering through an electromagnetic interaction, resulting in another fermion-antifermion pair.
        This is also known as a Drell-Yan process.
        At center-of-mass energies near and above the {\PZ} boson mass, {\PZ}-{\Pa} interference becomes nonnegligible.
      }\label{fig:drellYanDiagram}
  \end{center}
\end{figure}

\begin{table}[htbp]
  \begin{center}
    \caption[Everything in the universe]{
      The particles of the standard model, and some of their properties.
      All fermions have a corresponding antiparticle with opposite sign for all quantum numbers.
      Quarks and leptons are grouped by generation.
      Note that the listed $T^3$ applies only to left-handed fermions; right-handed fermions have $T^3=0$ and do not couple to the {\PWpm} (right-handed neutrinos, if they exist, do not couple to the {\PZ} either).
    }\label{tab:sm}
    \begin{tabular}{ccccc}
      \toprule % chktex 1
      Particle   & Mass ({\GeV})          & Charge ($e$) & $T^3$   & Gauge couplings \\
      \midrule
      \midrule
      \multicolumn{5}{c}{Scalar boson (spin 0)} \\
      \midrule
      {\PH}      & 125                    & 0            &         & {\PWpm}, {\PZ}  \\
      \midrule
      \midrule
      \multicolumn{5}{c}{Fermion (spin $1/2$)} \\
      \midrule
      {\Pqu}     & 0.023                  & $+2/3$       & $+1/2$  & {\Pg, \Pa, \PZ, \PWpm} \\
      {\Pqd}     & 0.048                  & $-1/3$       & $-1/2$  & {\Pg, \Pa, \PZ, \PWpm} \\
      \midrule
      {\Pe}      & $5.11 \times  10^{-4}$ & $-1$         & $+1/2$  & {\Pa, \PZ, \PWpm} \\
      {\Pne}     & $< 2.2 \times 10^{-9}$ & 0            & $-1/2$  & {\PZ, \PWpm}           \\
      \midrule
      {\Pqc}     & 1.28                   & $+2/3$       & $+1/2$  & {\Pg, \Pa, \PZ, \PWpm} \\
      {\Pqs}     & 0.95                   & $-1/3$       & $-1/2$  & {\Pg, \Pa, \PZ, \PWpm} \\
      \midrule
      {\Pm}      & $0.105$                & $-1$         & $+1/2$  & {\Pa, \PZ, \PWpm} \\
      {\Pnm}     & $< 1.7 \times 10^{-4}$ & 0            & $-1/2$  & {\PZ, \PWpm}           \\
      \midrule
      {\Pqt}     & 172                    & $+2/3$       & $+1/2$  & {\Pg, \Pa, \PZ, \PWpm} \\
      {\Pqb}     & 4.2                    & $-1/3$       & $-1/2$  & {\Pg, \Pa, \PZ, \PWpm} \\
      \midrule
      {\Pt}      & $1.77$                 & $-1$         & $+1/2$  & {\Pa, \PZ, \PWpm} \\
      {\Pnt}     & $< 0.018 $             & 0            & $-1/2$  & {\PZ, \PWpm}           \\
      \midrule
      \midrule
      \multicolumn{5}{c}{Vector boson (spin $1$)} \\
      \midrule
      {\Pg}      & 0                      & 0            & 0       & {\Pg}                  \\
      {\Pa}      & 0                      & 0            & 0       & {\PWpm}                \\
      {\PZ}      & 91.2                   & 0            & 0       & {\PWpm}                \\
      {\PWpm}    & 80.4                   & $\pm 1$      & $\pm 1$ & {\Pa, \PZ, \PWpm}      \\
      \bottomrule
    \end{tabular}
  \end{center}
\end{table}

There are six types of quarks which fall into three ``generations:'' up and down ({\Pqu} and {\Pqd}, first generation); charm and strange ({\Pqc} and {\Pqs}, second generation); and top and bottom ({\Pqt} and {\Pqb})\footnote{Top and bottom quarks are sometimes called truth and beauty by more romantic particle physicists.}.
Quark masses increase with each successive generation.
Up-type quarks ({\Pqu}, {\Pqc}, {\Pqt}) have electric charge $+2/3$ (in units of the positron charge) while down-type quarks have $-1/3$.
Quarks are the building blocks of hadron, including $\Pq\Paq'$ bound states called mesons and $\Pq\Pq'\Pq''/\Paq\Paq'\Paq''$ bound states called baryons, of which protons ($\Pqu\Pqu\Pqd$) and neutrons ($\Pqu\Pqd\Pqd$) are the most familiar.
Top quarks are too heavy to form bound states; they decay too quickly.
Hadrons are bound by the strong nuclear force, described by the theory of quantum chromodynamics (QCD).

The mediator for the strong force is the gluon, which like the photon is a massless spin-1 gauge boson.
The analog of electric charge is color charge, a notion originally introduced to explain how identical quarks could exist in the symmetric bound state of a hadron despite the Fermi exclusion principle~\cite{Griffiths:111880}.
Unlike electric charge, there are three types of color charge, typically called red, green, and blue, though these names are totally arbitrary\footnote{Negative color charges are typically called simply antired, antigreen and antiblue, but sometimes cyan, magenta and yellow, to continue the analogy with visible colors.}.
The analogy with color comes primary from the heuristic that natural states must ``colorless,'' i.e.\ a hadron may have equal parts color and corresponding anticolor as in a meson, but it may also be ``white,'' containing red, blue, and green in equal measures as in a baryon.
This propery, known as confinement, is why, for example, $\Pq\Pq\Paq$ bound states are not seen in nature.
It is also why a free quark has never been observed, and is not expected to be found, and why the strong interaction is short-range even though gluons are massless.

Confinement arises from the structure of QCD interactions and gluons themselves.
Among fermions, only quarks interact through the strong force, but gluons also carry color charge and interact with each other.
Because gluons interact with each other, do not have a distinct antiparticle, and are massless, they can split and radiate infinitely.
The resulting soft gluon interactions around quarks lead to an anti-screening effect that causes the strength of the strong force to change as a function of the distance between interacting quarks, with close quarks interacting less strongly as far as a single gluon exchange is concerned.
The origin of confinement is that as quark separation gets larger, the potential energy of strong interactions rises rapidly, until it is energetically favorable for the gluon connecting them to split into a $\Pq\Paq$ pair that screens them and effectively breaks off the interaction.
This enforces the requirement of colorless states: a single colored particle will cause more colored particles to be produced from vacuum until only colorless bound states remain.
This process is known as hadronization, and causes single quarks or gluons leaving a hard scattering interaction to make ``jets'' of many hadrons, each carrying a fraction of the original parton momentum, that enter the detector together.
Conversely, close-range QCD is relatively feeble, leading to ``asymptotic freedom,'' the property of partons within hadrons that they may be considered independent in high-energy collisions, because their interactions are weak enough that bound state effects may be neglected (see Section~\ref{sec:pp}).
Example Feynman diagrams for LO $\Pq\Paq \to \Pq\Paq\Pg$ scattering are shown in Fig.~\ref{fig:threeJet}.

\begin{figure}[htbp]
  \vspace{1em}
  \begin{center}
    \begin{fmffile}{threeJetS}
      \begin{fmfgraph*}(0.4,0.3) % chktex 36
        \fmfleft{d1,i1,i2,d2}
        \fmfright{d3,o2,o3,o1}
        \fmflabel{$\Pq$}{i1}
        \fmflabel{$\Paq$}{i2}
        \fmflabel{$\Pg$}{o1}
        \fmflabel{$\Paq$}{o2}
        \fmflabel{$\Pq$}{o3}
        \fmf{fermion}{i1,v1,i2}
        \fmf{gluon}{v1,v2}
        \fmf{phantom}{o2,v2,o3}
        \fmffreeze %chktex 1
        \fmf{fermion}{o2,v2,v3,o3}
        \fmffreeze %chktex 1
        \fmf{gluon,tension=0}{v3,o1}
      \end{fmfgraph*}
    \end{fmffile}
    \hspace{4em}
    \begin{fmffile}{threeJetT}
      \begin{fmfgraph*}(0.3,0.225) % chktex 36
        \fmfstraight %chktex 1
        \fmfleft{d0,i1,d1,i2}
        \fmfright{d2,o1,o2,o3}
        \fmflabel{$\Pq$}{i1}
        \fmflabel{$\Paq$}{i2}
        \fmflabel{$\Pg$}{o2}
        \fmflabel{$\Paq$}{o3}
        \fmflabel{$\Pq$}{o1}
        \fmf{fermion}{i1,v1,o1}
        \fmf{gluon}{v1,v2}
        \fmf{phantom}{o3,v2,i2}
        \fmffreeze %chktex 1
        \fmf{fermion}{o3,v3,v2,i2}
        \fmffreeze %chktex 1
        \fmf{gluon,tension=0}{v3,o2}
      \end{fmfgraph*}
    \end{fmffile}
    \vspace{1em}
    \caption[A three-jet Feynman diagram]{
        Example Feynman diagrams for tree-level $\Pq\Paq \to \Pq\Paq\Pg$ scattering.
        Several more diagrams also contribute to the process at LO\@; the final state gluon could be radiated by any of the four fermion lines, or the virtual gluon.
        Both outgoing quarks and the outgoing gluon carry color charge and each will therefore hadronize and enter the detector as a jet of many particles.
        In the cases shown, where the gluon is radiated by a final state quark, it is in general ambiguous whether it should be considered a final state particle when subsequently undergoes showering and hadronization, or whether it should be considered part of the quark's shower.
        This presents a significant challenge to theorists, as discussed in Section~\ref{sec:partonShower}.
      }\label{fig:threeJet}
  \end{center}
\end{figure}

Leptons may be electrically charged or neutral, and come in three generations, each containing one lepton of each type, a charged lepton and a corresponding neutrino.
In order of charged lepton mass, the generations are the electron and its neutrino ({\Pe} and {\Pne}), muon and its neutrino ({\Pm} and {\Pnm}), and tau and its neutrino ({\Pt} and {\Pnt}).
Taus decay quickly, with a mean lifetime of $2.9 \times 10^{-13}\unit{s}$ in their rest frame; muons also decay, but their lifetime ($2.2\unit{\mu s}$) is long compared to other time scales involved in particle collider experiments, so they are considered stable particles for the purposes of this work.
Neutrinos are known to have mass~\cite{Fukuda:1998mi,Ahmad:2001an,Ahmad:2002jz}, and the masses are known to be small but they have not been measured.
All leptons and quarks interact via the weak nuclear force, which is best known for causing the nuclear beta decay reaction $\Pn \to \Pp + \Pe^- + \Pane$.
Neutrinos are notable for coupling to the rest of the SM only through weak interactions, making them difficult to detect in practice.
Detectors at particle colliders make no attempt to detect neutrinos, and their involvement in the process is inferred only through the apparent momentum imbalance resulting from their absence.

The weak force operates through two mechanisms, charged-current and neutral-current interactions.
Neutral-current interactions proceed through exchange of a {\PZ} boson, an electrically neutral spin-1 mediator, and are analogous to electromagnetic interactions except for two important differences.
Unlike the {\Pa}, the {\PZ} has mass---in fact, one of the largest known masses at 91\GeV---giving it longitudinal polarization modes and limiting the range of the force because it decays with a halflife on the order of $10^{-25}\unit{s}$.
Also unlike QED, weak interactions do not respect parity (P) symmetry.
The {\PZ} boson couples more strongly to left-handed fermions (those with helicity opposite their direction of motion) and right-handed antiparticles than to their opposite-spin counterparts.
The degree of asymmetry varies by fermion type; notably, the {\PZ} does not couple at all to right-handed neutrinos.
Neutral-current interactions are still symmetric under combined charge conjugation (C) and parity (CP) transformations.
The neutral weak force leads to a Feynman diagram which looks exactly like that of Fig.~\ref{fig:drellYanDiagram}, which in fact implies that Feg.~\ref{fig:drellYanDiagram} is an oversimplification, or at least an approximation valid only for center-of-mass energies significantly below the {\PZ} boson mass, because the electromagnetic and neutral-current weak interactions interfere.

Charged-current interaction proceed through exchange of an electrically charged boson, the {\PWpm}, which has a mass around {80\GeV}.
Leptons couple to {\PWm} bosons in $\ell^-, \Panl$ pairs ({\PWp} bosons likewise with their antiparticles), causing {\Pm} and {\Pt} decays.
Lepton flavor in conserved in the sense that charged leptons couple to the {\PW} only in conjunction with the (anti-)neutrino from the same generation, so the total lepton number $N_\ell = n_{\ell^-} - n_{\ell^+} + n_{\Pnl} - n_{\Panl}$, where $n_\PX$ is the number of {\PX} particles in existence, is conserved separately for $\ell \in \left(\Pe, \Pm, \Pt \right)$. % chktex 36
Flavor conservation does not hold for quarks undergoing charged weak interactions.
An up-type quark always couples to the {\PW} in conjunction with a down-type quark, as it must to obey conservation of electric and color charge.
The pairings are in general described by a unitary $3 \times 3$ matrix known as the Cabibbo-Kobayashi-Maskawa (CKM) matrix which defines the inter-generational mixing.
This mixing allows heavy quarks to decay to lighter ones, and is thus responsible for the decay of hadrons that do not contain the $\Pq\Paq$ pair necessary for strong or electromagnetic decays.
The tree-level Feynman diagram for a leptonic top quark decay $\Pqt \to  \Pqb + \ell + \Panl$ is shown in Fig.~\ref{fig:topDecay}.

Charged-current interactions also do not respect parity symmetry, and in fact are maximally parity violating: the {\PW} couples only to left-handed fermions and right-handed antifermions.
Because neutrinos interact only through the weak force\footnote{Aside from gravity, presumably, but this interaction is not experimentally accessible and is not covered by the standard model.}, and neutral-current interactions also couple only to left-handed neutrinos, this implies that it is not clear if right-handed neutrinos even exist.
If they do, they have no way to interact with other matter and they are not part of the SM\@.
Unlike neutral-current interactions, charged-current interactions violate CP symmetry.
CP violation was first observed in neutral kaon mixing before the theory of the weak force was fully worked out~\cite{PhysRevLett.13.138}.
After flavor-changing charged currents were formalized it was realized that CP violation could arise from a complex phase in the CKM matrix, which arises in models with at least three generations of quarks\footnote{At the time, only the first two generations were known, so the observed CP asymmetry was taken as an early indication of the existence of top and bottom quarks.}~\cite{doi:10.1143/PTP.49.652}.
CP violation was subsequently confirmed by observation in a number of meson decays~\cite{AlaviHarati:1999xp,Fanti:1999nm,Aubert:2001sp,Abe:2001xe,Aaij:2012kz,Aaij:2013iua}

\begin{figure}[htbp]
  \vspace{1em}
  \begin{center}
    \begin{fmffile}{topDecay}
      \begin{fmfgraph*}(0.6,0.3) % chktex 36
        \fmfstraight %chktex 1
        \fmfleft{d1,i1,d2,d3}
        \fmfright{o1,d4,o2,o3}
        \fmflabel{$\Pqt$}{i1}
        \fmflabel{$\Pqb$}{o1}
        \fmflabel{$\Panl$}{o2}
        \fmflabel{$\ell$}{o3}
        \fmf{fermion}{i1,v1,o1}
        \fmf{boson,label={\PWp}}{v1,v2}
        \fmf{phantom}{d3,v2}
        \fmf{fermion}{o2,v2,o3}
      \end{fmfgraph*}
    \end{fmffile}
    \vspace{1em}
    \caption[Feynman diagram of a top quark decay]{
        Leading order Feynman diagram of a leptonic top quark decay $\Pqt \to \Pqb + \ell + \Panl$, which occurs via the charged-current weak interaction.
      }\label{fig:topDecay}
  \end{center}
\end{figure}



\section{Electroweak Symmetry Breaking and the Higgs Boson}
Everybody's favorite fundamental scalar boson



\section{Proton-Proton Collisions}\label{sec:pp}
Bang



\section{Diboson Physics}
Really only ZZ, but you get the point

\subsection{Vector Boson Scattering}
Scatter scatter


\section{Limitations and Possible Extensions}
It misses a few things

\subsection{Anomalous Gauge Couplings}
Pro tip: we won't see them



\section{Topics Covered In This Thesis}
