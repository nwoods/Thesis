\chapter{Object Reconstruction and Selection}

The raw detector information stored on disk after an event passes trigger selections is not yet suitable for physics analysis.
Hits in the tracker and muon systems, and energy deposits in the calorimeters, require significant processing to build physics objects that are interpretable in terms of the physics of the hard scatter.
Final state particles that interact with CMS are reconstructed from combinations of subdetector signals, final state particles are clustered into jets, charged particles are clustered by track origin to find proton-proton collision vertices, and visible particle momenta are summed to find the transverse momentum imbalance from undetectable particles (e.g.\ neutrinos).



\section{Track Reconstruction and Vertex Identification}
So many fits



\section{Particle Flow Reconstruction}
The overview

\subsection{PF Candidates}

\subsubsection{Muons}
yep

\subsubsection{Electrons and Charged Hadrons}
uh huh

\subsubsection{Photons and Neutral Hadrons}
yeah


\subsection{Jets}
sure


\subsection{Missing Transverse Energy}
ok



\section{Object Identification and Selection}
What to use in the actual analysis

\subsection{Electrons}
nice


\subsection{Muons}
even nicer


\subsection{Jets}
not as nice


\subsection{Final State Photon Radiation}
Bit of a mess


\subsection{Misidentified Objects}\label{sec:looseID}
Fake rates



\section{ZZ Candidate and Event Selection}
Explain the different classes of events (full spectrum, Higgs, on shell\ldots)

\subsection{Z Candidate Selection}\label{sec:zSelection}
Mass cuts and lepton pairing


\subsection{ZZ Candidate Selection}
Disambiguation for $>4$ leptons


\subsection{Background Estimation}


\subsection{VBS Signal Selection}\label{sec:vbsSelection}
Dijets and so on
