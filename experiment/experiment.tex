%!TEX root = ../nwoods_thesis.tex

\chapter{The CMS Experiment and the CERN LHC}
Production of controlled high-energy particle collisions, and detection of the decay products resulting from those collisions, are monumental technical challenges. The apparatus used to obtain the results presented in this thesis are the result of decades of work by thousands of scientists and engineers, with many of the techniques used developed in the course of building and operating previous experiments. The CERN Large Hadron Collider (LHC)~\cite{Evans:2008zzb,Bruning2012705} accelerates pairs of charged hadron (proton or lead ion) beams to high energies and collides them to provide a source of data to several fully independent detectors, including the Compact Muon Solenoid (CMS)~\cite{Chatrchyan:2008zzk}, which collected the data used in the studies presented here. Detailed descriptions of the LHC and CMS follow.


%-------------------------------------------------------------------------------
% LHC
%-------------------------------------------------------------------------------

\section{The CERN Large Hadron Collider}
The LHC, the most powerful particle accelerator and collider ever built, is a $26.7\unit{km}$ circumference ring of superconducting magnets running through tunnels roughly $100\unit{m}$ below the suburbs and countryside near Geneva, Switzerland. It first produced collisions suitable for collecting physics data in 2010 before generating large datasets with beam energies of $3.5\TeV$ in 2011 and $4\TeV$ in 2012. Following a shutdown for upgrades an repairs, it returned in 2015 and 2016 to deliver beam energies of $6.5\TeV$. Beams collide head-on so that the center-of-mass frame of the proton-proton system is the rest frame of the detectors, giving proton-proton center-of-mass energies of 7, 8, and $13\TeV$ respectively for collisions in 2011, 2012, and 2015--2016. Each successive energy was the highest ever acheived in controlled proton-proton collisions, giving unprecedented access to previously unobserved processes at every step.

In addition to increasing collision energies, the LHC increased its rate of collisions with each new machine configuration. The average event rate $dN / dt$ for a process with production cross section $\sigma$ is determined by the instantaneous luminosity $\lumiL$ of the collider,
\begin{equation}
  \frac{dN}{dt} = \lumiL\sigma
\end{equation}
so a high instantaneous luminosity is vital to the timely observation of rare processes like Higgs boson production. The LHC's unprecedented luminosities have allowed collection of the largest physics datasets in history.

\subsection{Layout and Accelerator Chain}
A chain of smaller accelerators repurposed from previous experiments feeds moderate-energy protons into LHC, where twin-bore superconducting magnets guide the beams around the ring in opposite directions and accelerate them to their final energy. The beams cross at four points around the ring and


\subsection{Design Goals and Constraints}
What they wanted\ldots


\subsection{Operation in 2015 and 2016}
\ldots and what we got



%-------------------------------------------------------------------------------
% CMS
%-------------------------------------------------------------------------------

\section{The Compact Muon Solenoid Detector}
It's big~\cite{Chatrchyan:2008zzk}

\subsection{Terminology and Geometry}
Something something definition of $\eta$ something


\subsection{Magnet and Inner Tracking System}

The magnet: how does it work?

\subsubsection{Pixel Detector}
Basically the same as your cameraphone

\subsubsection{Strip Tracker}
Silicon for days


\subsection{Electromagnetic Calorimeter}
Lots of crystals


\subsection{Hadronic Calorimeter}
More like HATECAL amirite?


\subsection{Muon Spectrometer}
There's also a return yoke

\subsubsection{Drift Tubes}
Are in the barrel

\subsubsection{Cathode Strip Chambers}
Are in the endcap

\subsubsection{Resistive Place Chambers}
Also exist


\subsection{Data Acquisition and Trigger}
Somethigng about DAQ

\subsubsection{Level-1 Trigger}
Obviously the best part

\subsubsection{High-Level Trigger}
Many Computers
