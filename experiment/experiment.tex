%!TEX root = ../nwoods_thesis.tex

\chapter{The CMS Experiment and the CERN LHC}
Production of controlled high-energy particle collisions, and detection of the decay products resulting from those collisions, are monumental technical challenges. The apparatus used to obtain the results presented in this thesis are the result of decades of work by thousands of scientists and engineers, with many of the techniques used developed in the course of building and operating previous experiments. The CERN Large Hadron Collider (LHC)~\cite{Evans:2008zzb,Bruning2012705} accelerates pairs of charged hadron (proton or lead ion) beams to high energies and collides them to provide a source of data to several fully independent detectors, including the Compact Muon Solenoid (CMS)~\cite{Chatrchyan:2008zzk}, which collected the data used in the studies presented here. Detailed descriptions of the LHC and CMS follow.


%-------------------------------------------------------------------------------
% LHC
%-------------------------------------------------------------------------------

\section{The CERN Large Hadron Collider}
The LHC, the most powerful particle accelerator and collider ever built, is a $26.7\unit{km}$ circumference ring of superconducting magnets running through tunnels roughly $100\unit{m}$ below the suburbs and countryside near Geneva, Switzerland.
It first produced collisions suitable for collecting physics data in 2010 before generating large datasets with beam energies of $3.5\TeV$ in 2011 and $4\TeV$ in 2012.
Following a shutdown for upgrades an repairs, it returned in 2015 and 2016 to deliver beam energies of $6.5\TeV$.
Beams collide head-on so that the center-of-mass frame of the proton-proton system is the rest frame of the detectors, giving proton-proton center-of-mass energies of 7, 8, and $13\TeV$ respectively for collisions in 2010--2011, 2012, and 2015--2016.
Each successive energy was the highest ever acheived in controlled proton-proton collisions, giving unprecedented access to previously unobserved processes at every step.

In addition to increasing collision energies, the LHC increased its rate of collisions with each new machine configuration.
The average event rate $\ddinline{N}{t}$ for a process with production cross section $\sigma$ is determined by the instantaneous luminosity $\lumiL$ of the collider,
\begin{equation}
  \label{eq:instLumi}
  \dd{N}{t} = \lumiL\sigma
\end{equation}
so a high instantaneous luminosity is vital to the timely observation of rare processes like Higgs boson production.
The LHC's unprecedented luminosities have allowed collection of the largest physics datasets in history.

The desire for high luminosities drove the decision to collide protons with other protons instead of with antiprotons as was done at Tevatron, LHC's predecessor at Fermilab in Batavia, IL\@.
Antiprotons simply cannot be produced in sufficient quantities for a collider on this scale.
Many of the physics processes Tevatron was designed to study are $\Pq\Paq$-initiated, so it is useful to have valence antiquarks available in the collisions.
The LHC was designed with Higgs boson production in mind, and the two most important Higgs production modes are proton/antiproton agnostic.
Even for $\Pq\Paq$-initiated processes, valence antiquarks are less critical at the LHC because, for equal parton momenta, protons have larger antiquark content at LHC energies than at Tevatron energies ($1.98\TeV$ center-of-mass energy) as discussed in Section~\ref{sec:pp}.

In addition to protons, the LHC can accelerate beams of lead nuclei to $2.51\TeV$ per nucleon, also the highest ever achieved.
All studies presented in this thesis were performed on proton-proton collision data, rendering the details of so-called ``heavy ion'' beams beyond the scope of this document.

Beams are maintained and manipulated with magnets, most of them made of superconducting NbTi winding cooled to $1.9\unit{K}$ by superfluid helium.
Dipole magnets with fields up to $8.33\unit{T}$ bend the beam around the ring, interspersed with quadrupoles for focusing.
More quadrupoles and higher-moment magnets apply a number of corrections and squeeze the beams for collisions.
Superconducting radio frequency (RF) cavities operating at $400\unit{MHz}$ accelerate the beam, maintain it at its final energy, and maintain bunch shape and spacing.


\subsection{Accelerator Chain, Layout, and Detectors}
The LHC was built in tunnels originally constructed for the Large Electron-Positron Collider (LEP), an $\Pe^+$-$\Pe^-$ collider that operated from 1989 to 2000.
Using existing caverns, tunnels, and infrastructure was a substantial cost-saving measure, but imposed several important constraints on the LHC's design.
In LEP, the electron and positron beams could be accelerated in opposite directions by the same magnets, because they are oppositely charged.
Conversely, proton beams require opposite magnetic fields for the two beams.
Because the tunnels were not wide enough to accommodate two completely separate beam lines, most of the magnets in the LHC use a twin-bore design in which the pipes and windings for the two beams share a common cryogenic system.
The electromagnetic, mechanical, and cryogenic coupling of the two beamlines represents a significant engineering challenge.

Because no single accelerator has the dynamic range necessary to take a stationary proton to \TeV-scale energies, a chain of smaller accelerators repurposed from previous experiments feeds moderate-energy protons into LHC\@.
Protons are obtained by ionizing hydrogen atoms, then accelerated to $50\MeV$ by the Linac~2 linear accelerator and injected into the Proton Syncrotron Booster~(PSB), the first of several circular accelerators.
The PSB feeds $1.4\GeV$ protons into the Proton Synchrotron~(PS), which in turn injects them into the Super Proton Synchrotron~(SPS) at $26\GeV$.
The protons are then accelerated to $450\GeV$ in the SPS before being injected into LHC\@.

The ring is divided into eight sectors, each of which features a $528\unit{m}$ straight section connected to the adjacent sections by $2.45\unit{km}$ arcs.
The straight section length was set by the need for RF cavities to accelerate LEP beams to counteract synchrotron radiation, which is a primary factor limiting electron and positron beam energy.
This is not ideal for proton beams; protons' much higher mass means they radiate less and need fewer RF cavities.
The straight sections feature ``insertion'' points numbered with Point~1 at the main CERN site in Meyrin, Switzerland, and the rest numbered~2--8, increasing in the clockwise direction when viewed from above.
Points~1, 2, 5, and~8 have beam crossing points and host detectors to study the resulting proton-proton collisions.
Points~3 and~7 feature collimators to remove nonuniformities in the beams.
The RF cavities are at Point~4 and the beams are dumped after use into absorbers at Point 6\@.

The CMS detector is at Point~5 in Cessy, France, the furthest point on the ring from the Meyrin site and Point~1, which houses ATLAS~\cite{Aad:2008zzm}, a similar but fully independent general-purpose particle detector.
CERN and the science funding agencies support CMS and ATLAS equally so that any measurement or discovery made by one can be made concurrently or verified by the other.
The other two experimental insertions feature specialized detectors studying collisions at lower-luminosity beam interaction points.
The LHCb detector~\cite{Alves:2008zz}, at Point~8, studies hadronic physics with an emphasis on b-mesons, and ALICE~\cite{Aamodt:2008zz} studies heavy ion collisions at Point~2\@.
Three smaller experiments share interaction points with the larger detectors, with TOTEM~\cite{Anelli:2008zza} studying proton structure and the total proton-proton interaction cross section next to CMS\@; LHCf~\cite{Adriani:2008zz} studying the $\pi^0$ energy spectrum and multiplicity near ATLAS\@; and MoEDAL~\cite{Acharya:2014nyr} searching for magnetic monopoles or other heavy, stable, ionizing particles at Point~8 with LHCb.



\subsection{Operating Parameters}
With the beam energy set by the radius of the ring and the strength of available magnets, the number of interesting physics events produced in LHC collisions depends only on the integrated luminosity
\begin{equation}
  \lumiL_\textit{int} = \int \lumiL \cmsSymbolFace{d}t,
\end{equation}
where $\lumiL$ is the instantaneous luminosity defined in Eq.~\ref{eq:instLumi} and the integral runs over the time the machine spends in collisions mode.
LHC's availability for collisions depends on the electrical and mechanical stability of the accelerators and their support systems, including the cryogenics and the vacuum in the beam pipe.
The instantaneous luminosity while running depends only on the beam parameters.
For symmetric beams which each have $n_b$ colliding gaussian bunches of intensity (i.e.\ number of protons in the bunch) $N_b$, orbiting the ring with frequency $f_\textit{rev}$ and relativistic factor $\gamma=E_p/m_p$, the instantaneous luminosity is give by
\begin{equation}
  \lumiL = f_\textit{rev} \frac{n_b N_b^2 \gamma}{4\pi \beta^\ast \epsilon_N} R,
\end{equation}
where $\beta^\ast$ is the amplitude of the beams' betatron oscillations around the nominal ring path at the interaction point, the normalized emittance $\epsilon_N$ is a measure of the beams' spread in both position and momentum space, and $R$ is a geometrical factor accounting for the beam crossing angle,
\begin{equation}
  R = \sqrt{1 + \left(\frac{\theta_c \sigma_z}{2\sigma^\ast}\right)^2}.
\end{equation}
Here $\theta_c$ is the beams' crossing angle, and $\sigma_z$ and $\sigma^\ast$ are respetively the longitudinal and transverse RMS widths of the bunches in the lab frame.


\subsubsection{Design}
The machine parameters in the LHC design specification can be seen in the first column of Table~\ref{tab:lhcparams}.
Many of the design parameters, in particular the energy and number of colliding bunches, have not been met due to a failure during initial testing in September 2008.
A fault in a superconducting bus bar connection between a dipole and a quadrupole caused an electrical arc which ruptured the cryostat, leading to a rapid and destructive reslease of helium gas~\cite{Bajko:1168025}.
When LHC was brought back online in 2010 after repairs and upgrades intended to prevent similar incidents in the future, its operating parameters were changed to reduce the risk of future incidents, with some luminosity-related parameters adjusted to mitigate the resulting loss of physics discovery potential as much as possible.
Further upgrades have improved machine performance, with some parameters now meeting or exceeding the design specification.


\begin{table}
\caption{
  LHC beam parameters as designed and in practice.
  As stated in the text, $n_b$ is the number of colliding bunches, $N_b$ is the number of protons in each bunch, $\beta^\ast$ is the betatron amplitude at the interaction point, $\epsilon_N$ is the normalized emittance, and $\lumiL_{\left(\textit{int}\right)}$ is the instantaneous (integrated) luminosity.
  }
\centering
\begin{tabular}{lllllll}
\hline
                                                                     & \multicolumn{1}{c}{Design} & \multicolumn{3}{c}{Run I}  & \multicolumn{2}{c}{Run II} \\
\hline
Year                                                                 &          & 2010    & 2011    & 2012    & 2015    & 2016                               \\
\hline\hline
Energy per beam $\left(\TeVns\right)$                                & 7        & 3.5     & 3.5     & 4       & 6.5     & 6.5                                \\
Bunch spacing $\left(\unitns{ns}\right)$                             & 25       & 150     & 50      & 50      & 25      & 25                                 \\
$n_b$                                                                & 2808     & 348     & 1331    & 1368    & 2232    & 2208                               \\
$N_b \left(10^{11}\right)$                                           & 1.15     & 1.2     & 1.5     & 1.7     & 1.15    & 1.25                               \\
$\beta^\ast \left(\unitns{m}\right)$                                 & 0.55     & 3.5     & 1.0     & 0.6     & 0.8     & 0.4                                \\
$\epsilon_N \left(\unitns{mm}\unit{mrad}\right)$                     & 3.75     & 2.2     & 2.3     & 2.5     & 3.5     & 3.0                                \\
Peak pileup                                                          & FIXME    & 4       & 17      & 37      & 22      & 49                                 \\
Peak $\lumiL \left(10^{34}\unitns{cm}^{-2}\unitns{s}^{-1}\right)$    & 1        & 0.02    & 0.35    & 0.77    & 0.52    & 1.53                               \\
$\lumiL_\textit{int} \left(\fbinvns\right)$                          &          & 0.04    & 6.1     & 23.3    & 4.2     & 41.1                               \\
\hline

\end{tabular}
\label{tab:lhcparams}
\end{table}


\subsubsection{Run I}
The LHC was brought back online in 2010 at half its design energy, with $3.5\TeV$ beam energy, which was increased to $4\TeV$ in 2012. The bunch intensity was also lower, with bunches spaced $50\unit{ns}$ apart instead of $25\unit{ns}$.
The longer bunch spacing was chosen to allow full exploitation of excellent injection chain performance~\cite{1742-6596-455-1-012001}.
Beams exiting the SPS had bunch intensity as much as 50\% higher than anticipated in the original LHC design and beam emittance as low as 67\% of nominal.
This allowed the LHC to achieve 77\% of its design instantaneous luminosity in 2012 despite having roughly half as many bunches in each beam.

Machine availability was overall good considering the complexity and relative newness of the LHC, with about 36\% of scheduled time spent in stable beams.
In all, LHC delivered $6.1\fbinv$ to CMS and ATLAS in 2011 and $23.3\fbinv$ in 2012, enough to allow discovery of the Higgs boson.


\subsubsection{Run II}
The LHC shut down for 2013 and 2014 to allow a number of repairs and upgrades, including measurements, repairs and upgrades on the electrical connections and cryogenic safety systems like the ones that failed in 2008\@.
Beam energies could then be increased to $6.5\TeV$, close to the nominal $7\TeV$.
The bunch spacing was decreased to $25\unit{ns}$ while maintaining low emittance and high bunch intensity with the implementation of the beam compression merging and splitting (BCMS) scheme in which bunches are merged in the PS before they are split for injection into SPS, allowing higher bunch intensity~\cite{Papaphilippou:2014qwa}.
This was offset by vacuum problems in the SPS beam dump, which limited the total number of colliding bunches to around 2200~\cite{Frederick:2235979}.
Improvements in collimators and beam optics reduced $\beta^\ast$ to $40\unit{cm}$ in 2016, lower than the design $\beta^\ast$ of $55\unit{cm}$.
Overall instantaneous luminosities were substantially higher than originally designed.

Machine availability in Run II was excellent, with over 60\% of planned time spent in stable beams~\cite{Frederick:2235979}.
Mechanical problems kept LHC out of commission for much of 2015, and only $4.2\fbinv$ were delivered to Points~1 and~5, but the integrated luminosity in 2016, $41.1\fbinv$, was far above the roughly $25\fbinv$ expected and more than all previous runs combined.



%-------------------------------------------------------------------------------
% CMS
%-------------------------------------------------------------------------------

\section{The Compact Muon Solenoid Detector}
The CMS detector~\cite{Chatrchyan:2008zzk} is a general-purpose particle detector located in a cavern roughly $100\unit{m}$ below LHC Point~5.
Though designed to do a wide range of physics analyses, CMS was designed specifically with Higgs boson discovery in mind.
Primary design goals include
\begin{itemize}
  \item High-efficiency reconstruction of charged particles with precise measurement of their trajectories and momenta
  \item Good electromagnetic energy resolution, including diphoton and dielectron mass resolution
  \item Hermetic calorimetry for good missing transverse energy and dijet mass resolution
  \item Good muon identification, momentum resolution (including dimuon mass resolution), and charge determination over a broad range of energies
\end{itemize}
To this end, CMS features a silicon tracker, a scintillating crystal electromagnetic calorimeter (ECAL), and a hermetic hadronic calorimeter (HCAL) inside a $3.8\unit{T}$ solenoid magnet surrounded by ionized gas muon tracking devices. Decisions on which events to read out are made on-line by a two-level trigger system.
Descriptions of these systems follow.


\subsection{Terminology and Geometry}
The CMS detector systems are arranged in cylindrical layers with the interaction point at the center, serving as the origin for the coordinate system.
The coordinate system is defined with the positive-$x$ direction pointing toward the center of the ring, positive-$y$ pointing vertically up, and positive-$z$ pointing parallel to the beam in the counterclockise direction when the LHC ring is viewed from above.
Particle momenta are typically expressed in quasicylindrical coordinates $\left(\pt,\eta,\phi\right)$.
Here $\pt$ is the magnitude of the particle's momentum transverse to the beam
\begin{equation}
  \pt \equiv \sqrt{p_x^2 + p_y^2},
\end{equation}
and $\phi$ is the azimuthal angle, i.e.\ the angle from the $x$-axis to the particle's trajectory in the $x$-$y$ plane.
The pseudorapidity~$\eta$ is defined as
\begin{equation}
  \eta \equiv -\ln \left[\tan\left(\frac{\theta}{2}\right)\right]
\end{equation}
where $\theta$ is the polar angle measured from the $z$-axis.
In the limit of massless particles, the pseudorapidity is the same as the rapidity
\begin{equation}
  y \equiv \frac{1}{2} \ln \left(\frac{E+p_z}{E-p_z}\right).
\end{equation}
Pseudorapidity is preferred to rapidity because it is purely geometrical, with no dependence on the particle energy.
Both are preferred over $\theta$ because rapidity differences are invariant under longitudinal boosts, and because hadron flux at colliders is roughly constant as a function of rapidity.
The transverse energy~$\et$ is the the magnitude of the particle's four-momentum transverse to the beam, equal to $\pt$ in the limit of massless particles.
Spatial coordinates are expressed as $\left(r,\eta,\phi\right)$, where $r$ is the distance from the beam in the $x$-$y$ plane.


\subsection{Magnet and Inner Tracking System}
A particle of charge $q$ moving through a uniform magnetic field of strength $B$ that points in the $z$ direction will travel in a helix of radius $R$, given by
\begin{equation}
  R = \frac{\pt}{\lvert q\rvert B}
\end{equation}
with the chirality of the helix determined by the sign of $q$.
Thus one can determine the transverse momentum of the particle by measuring its path through the magnetic field and finding the radius of curvature.
In practice, all but the lowest-energy particles leave too short an arc in the detector for direct measurement of the radius, so the sagitta~$s$ of the arc is used instead, given by
\begin{equation}
  s = \frac{qBL^2}{8\pt}
\end{equation}
where $L$ is the length of the chord spanning the arc (typically equal to the radius of the tracking system).
The transverse momentum resolution varies as
\begin{equation}
  \frac{\delta \pt}{\pt} \propto \frac{\pt}{BL^2},
\end{equation}
so a strong field and a large tracking volume are vital to keeping measurements precise even at high energies.

To this end, CMS contains the world's largest superconducting magnet\footnote{Largest in the sense of having the largest stored energy when at constant full field. The largest by size is the ATLAS barrel toroid.}, a solenoid $13\unit{m}$ long and $6\unit{m}$ in diameter, which generates a nearly-uniform $3.8\unit{T}$ field in the centralmost part of the detector.
To measure the paths of charged particles in the field, the volume closest to the interaction point contains layers of silicon sensors that detect hits from charged particles with high efficiency and excellent position resolution, between $4.4\unit{cm}$ and $1.1\unit{m}$ from the beam for $2.7\unit{m}$ on either side of the interaction point.
This system, called the inner tracker, consists of an inner pixel detector surrounded by a larger silicon strip detector.
Both consist of concentric cylinders of sensors covering the barrel of the detector capped by discs covering the high-$\eta$ region, up to $\abseta < 2.5$.
With a total of roughly $200\unit{m}^2$ of silicon, the inner tracker is the largest silicon tracker in the world, which means that tracks may be reconstructed with hits in as many as 14 layers.
The downside of this is that the tracker represents a substantial amount of material for electrons and photons to interact with before they reach the calorimeters, with total material budget between 0.4 radiation lengths ($\eta=0$) and 1.8 radiation lengths ($\abseta \approx 1.4$).
The tracker-only $\pt$ uncertainty is around~1.2\% at $200\GeV$ and 15\% at~$1\TeV$.


\subsubsection{Pixel Detector}
The pixel detector, consisting of three layers in the barrel and two in the endcap, is responsible for accurate reconstruction of primary proton-proton interaction vertices and secondary vertices from $\Pqb$-meson decays, as well as providing ``seed'' tracks that may be used in strip tracker reconstruction.
As the system closest to the interaction point, the pixel system experiences the highest charged-particle flux must have extremely fine granularity to differentiate between nearby particles.
The 66~million pixels in the system have a cell size of $100 \times 150\unit{\mu m}^2$.
Interpolation of the analog signals from the individual pixels allows a final spatial resolution of $15\unit{\mu m}$ in each direction.
The outermost barrel layer is $10.2\unit{cm}$ from the beam, and the second endcap disk is $46.5\unit{cm}$ from the interaction point.
The sensor modules are arranged such that at least three sensors cover the solid angle within the pixels' acceptance.


\subsubsection{Strip Tracker}
Outside the pixels is the silicon strip tracker, extending out to $1.1\unit{m}$ in the $r$ direction and $\pm 2.8\unit{m}$ in the $z$ direction.
The tracker is divided into inner and outer subdetectors, each of which has both barrel cylinders and endcap discs.
In total, there are ten layers in the barrel and nine in each of the endcaps.
The inner tracker uses $320\unit{\mu m}$-thick sensors with a typical strip cell size of $10\unit{cm} \times 80\unit{\mu m}$, leading to hit resolutions of 23--$35\unit{\mu m}$.
The outer tracker uses $500\unit{\mu m}$-thick sensors with typical strip sizes up to $25\unit{cm} \times 180\unit{\mu m}$, leading to hit resolutions of 35--$53\unit{\mu m}$.


\subsection{Electromagnetic Calorimeter}
Outside of the tracker is the electromagnetic calorimeter (ECAL), which is designed to absorb and measure the energy of electrons and photons.
ECAL is made of 68,524 lead tungstate (\pbwo) crystals arranged in a cylindrical barrel covering $\abseta < 1.444$ and two endcap discs covering $1.566 < \abseta < 3.0$.
\pbwo{} is 


\subsection{Hadronic Calorimeter}
More like HATECAL amirite?


\subsection{Muon Spectrometer}
There's also a return yoke

\subsubsection{Drift Tubes}
Are in the barrel

\subsubsection{Cathode Strip Chambers}
Are in the endcap

\subsubsection{Resistive Place Chambers}
Also exist


\subsection{Data Acquisition and Trigger}
Somethigng about DAQ

\subsubsection{Level-1 Trigger}
Obviously the best part

\subsubsection{High-Level Trigger}
Many Computers


\subsection{Luminosity Determination}
BRIL and whatnot
