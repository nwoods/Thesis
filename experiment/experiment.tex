%!TEX root = ../nwoods_thesis.tex

\chapter{The CMS Experiment and the CERN LHC}
Production of controlled high-energy particle collisions, and detection of the decay products resulting from those collisions, are monumental technical challenges. The apparatus used to obtain the results presented in this thesis are the result of decades of work by thousands of scientists and engineers, with many of the techniques used developed in the course of building and operating previous experiments. The CERN Large Hadron Collider (LHC)~\cite{Evans:2008zzb,Bruning2012705} accelerates pairs of charged hadron (proton or lead ion) beams to high energies and collides them to provide a source of data to several fully independent detectors, including the Compact Muon Solenoid (CMS)~\cite{Chatrchyan:2008zzk}, which collected the data used in the studies presented here. Detailed descriptions of the LHC and CMS follow.


%-------------------------------------------------------------------------------
% LHC
%-------------------------------------------------------------------------------

\section{The CERN Large Hadron Collider}
The LHC, the most powerful particle accelerator and collider ever built, is a $26.7\unit{km}$ circumference ring of superconducting magnets running through tunnels roughly $100\unit{m}$ below the suburbs and countryside near Geneva, Switzerland.
It first produced collisions suitable for collecting physics data in 2010 before generating large datasets with beam energies of $3.5\TeV$ in 2011 and $4\TeV$ in 2012.
Following a shutdown for upgrades an repairs, it returned in 2015 and 2016 to deliver beam energies of $6.5\TeV$.
Beams collide head-on so that the center-of-mass frame of the proton-proton system is the rest frame of the detectors, giving proton-proton center-of-mass energies of 7, 8, and $13\TeV$ respectively for collisions in 2010--2011, 2012, and 2015--2016.
Each successive energy was the highest ever acheived in controlled proton-proton collisions, giving unprecedented access to previously unobserved processes at every step.

In addition to increasing collision energies, the LHC increased its rate of collisions with each new machine configuration.
The average event rate $\ddinline{N}{t}$ for a process with production cross section $\sigma$ is determined by the instantaneous luminosity $\lumiL$ of the collider,
\begin{equation}
  \dd{N}{t} = \lumiL\sigma
\end{equation}
so a high instantaneous luminosity is vital to the timely observation of rare processes like Higgs boson production.
The LHC's unprecedented luminosities have allowed collection of the largest physics datasets in history.

The desire for high luminosities drove the decision to collide protons with other protons instead of with antiprotons as was done at Tevatron, LHC's predecessor at Fermilab in Batavia, IL\@.
Antiprotons simply cannot be produced in sufficient quantities for a collider on this scale.
Many of the physics processes Tevatron was designed to study are $\Pq\Paq$-initiated, so it is useful to have valence antiquarks available in the collisions.
The LHC was designed with Higgs boson production in mind, and the two most important Higgs production modes are proton/antiproton agnostic.
Even for $\Pq\Paq$-initiated processes, valence antiquarks are less critical at the LHC because, for equal parton momenta, protons have larger antiquark content at LHC energies than at Tevatron energies ($1.98\TeV$ center-of-mass energy) as discussed in Section~\ref{sec:pp}.

In addition to protons, the LHC can accelerate beams of lead nuclei to $2.51\TeV$ per nucleon, also the highest ever achieved.
All studies presented in this thesis were performed on proton-proton collision data, rendering the details of so-called ``heavy ion'' beams beyond the scope of this document.

Beams are maintained and manipulated with magnets, most of them made of superconducting NbTi winding cooled to $1.9\unit{K}$ by superfluid helium.
Dipole magnets with fields up to $8.33\unit{T}$ bend the beam around the ring, interspersed with quadrupoles for focusing.
More quadrupoles and higher-moment magnets apply a number of corrections and squeeze the beams for collisions.
Superconducting radio frequency (RF) cavities operating at $400\unit{MHz}$ accelerate the beam, maintain it at its final energy, and maintain bunch shape and spacing.


\subsection{Accelerator Chain, Layout, and Detectors}
The LHC was built in tunnels originally constructed for the Large Electron-Positron Collider (LEP), an $\Pe^+$-$\Pe^-$ collider that operated from 1989 to 2000.
Using existing caverns, tunnels, and infrastructure was a substantial cost-saving measure, but imposed several important constraints on the LHC's design.
In LEP, the electron and positron beams could be accelerated in opposite directions by the same magnets, because they are oppositely charged.
Conversely, proton beams require opposite magnetic fields for the two beams.
Because the tunnels were not wide enough to accommodate two completely separate beam lines, most of the magnets in the LHC use a twin-bore design in which the pipes and windings for the two beams share a common cryogenic system.
The electromagnetic, mechanical, and cryogenic coupling of the two beamlines represents a significant engineering challenge.

Because no single accelerator has the dynamic range necessary to take a stationary proton to \TeV-scale energies, a chain of smaller accelerators repurposed from previous experiments feeds moderate-energy protons into LHC\@.
Protons are obtained by ionizing hydrogen atoms, then accelerated to $50\MeV$ by the Linac~2 linear accelerator and injected into the Proton Syncrotron Booster~(PSB), the first of several circular accelerators.
The PSB feeds $1.4\GeV$ protons into the Proton Synchrotron~(PS), which in turn injects them into the Super Proton Synchrotron~(SPS) at $26\GeV$.
The protons are then accelerated to $450\GeV$ in the SPS before being injected into LHC\@.

The ring is divided into eight sectors, each of which features a $528\unit{m}$ straight section connected to the adjacent sections by $2.45\unit{km}$ arcs.
The straight section length was set by the need for RF cavities to accelerate LEP beams to counteract synchrotron radiation, which is a primary factor limiting electron and positron beam energy.
This is not ideal for proton beams; protons' much higher mass means they radiate less and need fewer RF cavities.
The straight sections feature ``insertion'' points numbered with Point~1 at the main CERN site in Meyrin, Switzerland, and the rest numbered~2--8, increasing in the clockwise direction when viewed from above.
Points~1, 2, 5, and~8 have beam crossing points and host detectors to study the resulting proton-proton collisions.
Points~3 and~7 feature collimators to remove nonuniformities in the beams.
The RF cavities are at Point~4 and the beams are dumped after use into absorbers at Point 6\@.

The CMS detector is at Point~5 in Cessy, France, the furthest point on the ring from the Meyrin site and Point~1, which houses ATLAS, a similar but fully independent general-purpose particle detector.
CERN and the science funding agencies support CMS and ATLAS equally so that any measurement or discovery made by one can be made concurrently or verified by the other.
The other two experimental insertions feature specialized detectors studying collisions at lower-luminosity beam interaction points.
The LHCb detector, at Point~8, studies hadronic physics with an emphasis on b-mesons, and ALICE studies heavy ion collisions at Point~1\@.
Three smaller experiments share interaction points with the larger detectors, with TOTEM studying proton structure and the total proton-proton interaction cross section next to CMS, LHCf studying the $\pi^0$ energy spectrum and multiplicity near ATLAS, and MoEDAL searching for magnetic monopoles or other heavy, stable, ionizing particles at Point~8 with LHCb.



\subsection{Performance Goals and Constraints}
What they wanted\ldots


\subsection{Operation in 2015 and 2016}
\ldots and what we got



%-------------------------------------------------------------------------------
% CMS
%-------------------------------------------------------------------------------

\section{The Compact Muon Solenoid Detector}
It's big~\cite{Chatrchyan:2008zzk}

\subsection{Terminology and Geometry}
Something something definition of $\eta$ something


\subsection{Magnet and Inner Tracking System}

The magnet: how does it work?

\subsubsection{Pixel Detector}
Basically the same as your cameraphone

\subsubsection{Strip Tracker}
Silicon for days


\subsection{Electromagnetic Calorimeter}
Lots of crystals


\subsection{Hadronic Calorimeter}
More like HATECAL amirite?


\subsection{Muon Spectrometer}
There's also a return yoke

\subsubsection{Drift Tubes}
Are in the barrel

\subsubsection{Cathode Strip Chambers}
Are in the endcap

\subsubsection{Resistive Place Chambers}
Also exist


\subsection{Data Acquisition and Trigger}
Somethigng about DAQ

\subsubsection{Level-1 Trigger}
Obviously the best part

\subsubsection{High-Level Trigger}
Many Computers
