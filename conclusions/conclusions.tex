%!TEX root = ../nwoods_thesis.tex

\chapter{Conclusions}\label{ch:conclusions}

\section{Summary}

Diboson studies in high-energy particle collisions are important probes of the electroweak sector of the standard model.
In particular, measurements of resonant and nonresonant four-lepton production shed light on the couplings between the neutral gauge bosons and on the details of electroweak symmetry breaking.
Decays to four charged leptons (electrons or muons) are rare, but they can be easily identified and fully reconstructed, and they represent a clean channel with low backgrounds.
The LHC at CERN has generated proton-proton collisions with a center-of-mass energy of {13\TeV} at record-breaking luminosities, providing an unprecedented opportunity to study such processes.
The CMS detector is well designed for these measurements and collected a high-quality dataset corresponding to an integrated luminosity of {35.9\ifb}.
Several studies of four-lepton production were performed with this dataset and reported here.

Because the four-lepton channel is so clean, event selections are loose and efficiencies are high.
Most backgrounds are estimated from data.
The full four-lepton spectrum includes resonant single-{\PZ} decays ($\pp \to \PZ \to 4\ell$), resonant Higgs boson decays ($\pp \to \PH \to 4\ell$), continuum production of a single {\PZ} boson and a nonresonant lepton pair ($\pp \to \PZ\Pa^\ast \to 4\ell$), and continuum production of two on-shell ($60 < m_{\ell\ell} < 120\GeV$) {\PZ} bosons ($\pp \to \ZZ \to 4\ell$).

Both inclusive and differential {\ZZ} cross sections were measured.
Inclusive cross sections were measured with a signal strength found by a maximum likelihood fit.
The measured fiducial {\ZZ} cross section, subject to the requirements of Table~\ref{tab:fiducialDefs}, is
\begin{equation}
  \sigma_\text{fid} (\pp \to \ZZ \to 4\ell) = 40.9 \pm 1.3 \stat \pm 1.4 \syst \pm 1.0 \lum \unit{fb}.
\end{equation}
The total {\ZZ} cross section, subject only to the constraint that both {\PZ} bosons be on-shell, was extrapolated with an acceptance correction and combined with the smaller ({2.9\ifb}) dataset collected in 2015\@.
Its measured value is
\begin{equation}
  \sigma(\pp \to \ZZ) = 17.2 \pm 0.5 \stat \pm 0.7 \syst \pm 0.4 \thy \pm 0.4 \lum \unit{pb}.
\end{equation}
The {\Zfourl} branching fraction was measured for events with $80 < m_{4\ell} < 100\GeV$ and a requirement of $m_{\ell\ell} > 4\GeV$ for all opposite-sign, same-flavor lepton pairs, and found to be
\begin{equation}
  \mathcal{B}\left(\Zfourl\right) = 4.8 \pm 0.2 \stat \pm 0.2 \syst \pm 0.1 \thy \pm 0.1 \lum \times 10^{-6}.
\end{equation}
Differential cross sections were measured as functions of a number of observables including fully leptonic kinematic variables and quantities related to the production of associated jets.
All results are in agreement with SM predictions.

A search was performed for fully electroweak {\ZZjj} production using a gradient-boosted decision tree.
An excess consistent with VBS was found at the level of 2.7 standard deviations above the null hypothesis ($1.6\sigma$ expected).
This corresponds to a measured electroweak fiducial cross section of
\begin{equation}
  \sigma_\text{fid}(\pp \to \PZ\PZ\Pj\Pj\text{(EWK)} \to 4\ell\Pj\Pj) = 0.40^{+0.21}_{-0.16} \stat ^{+0.13}_{-0.09} \syst \unit{fb},
\end{equation}
consistent with SM predictions.

Searches were performed for anomalous triple and quartic gauge couplings.
The aTGC search, considered in an effective lagrangian framework, used the invariant mass of inclusive {\ZZ} events to set the most stringent 95\% CL limits to date on {\ZZZ} and {$\ZZ\Pa$} couplings,
\begin{equation}
  \begin{aligned}
  & -0.0012 < f_4^\PZ < 0.0010   ,  & -0.0010 < f_5^\PZ < 0.0013 , \\
  & -0.0012 < f_4^\Pa < 0.0013   ,  & -0.0012 < f_5^\Pa < 0.0013 .
  \end{aligned}
\end{equation}
Two-dimensional limits were also set.
The aQGC search, performed in an effective field theory parameterization with {\ZZjj} events, set the most stringent 95\% CL limits to date on several dimension-8 operators which govern quartic gauge couplings,
\begin{equation}
  \begin{aligned}
    &-0.46  &<& \quad  f_\text{T0} / \Lambda^4  &<&  \quad  0.44  & \TeV^{-4}, \\
    &-0.61  &<& \quad  f_\text{T1} / \Lambda^4  &<&  \quad  0.61  & \TeV^{-4}, \\
    &-1.2   &<& \quad  f_\text{T2} / \Lambda^4  &<&  \quad  1.2   & \TeV^{-4}, \\
    &-0.84  &<& \quad  f_\text{T8} / \Lambda^4  &<&  \quad  0.84  & \TeV^{-4}, \\
    &-1.8   &<& \quad  f_\text{T9} / \Lambda^4  &<&  \quad  1.8   & \TeV^{-4}.
  \end{aligned}
\end{equation}



\section{Outlook}

Diboson measurements have great potential to shed further light on the SM or find deviations from it.
In the long term, cross section measurements at higher center-of-mass energies are of great interest because new physics should be most obvious at high $\sqrt{s}$.
With no new colliders expected in the near future\footnote{The LHC may operate at $\sqrt{s} = 14\TeV$ in the near future, which would be useful but only marginally more likely to reveal new physics, in line with the marginal increase in energy.}, progress will first come in the form of precision measurements of processes that are in principle accessible now.
The statistical uncertainties of the measured inclusive cross sections are now comparable to or smaller than the systematic uncertainties, and systematics should dominate after the inclusion of data collected in 2017, even if new techniques are developed which reduce lepton efficiency and trigger uncertainties somewhat.
Uncertainty reductions that can be expected in the short term will make the inclusive measurements somewhat more useful---experimental uncertainties will be comparable to or smaller than theoretical uncertainties, allowing further theoretical improvements to be of interest---but the improvements will overall be limited by the difficulty of reducing systematics in the highly active environments envisioned for future LHC runs.

Differential cross sections and searches, however, will be statistically limited for some time and will benefit greatly from additional luminosity at $\sqrt{s} = 13\TeV$.
Statistical uncertainties dominate in almost all bins in the differential cross sections.
Assuming no deviations from the SM, data collected in 2017 will likely be enough to allow $3\sigma$ evidence for VBS, when added to the 2016 data presented here.
The statistical power for the aGC searches comes largely from the highest-mass bins, where very few events have been observed---only three above {800\GeV} and none above {1\TeV}, even in the inclusive selection.
Further data collection will improve these limits substantially and place stringent restrictions on BSM neutral gauge boson couplings---or discover them.
