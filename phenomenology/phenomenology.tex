%!TEX root = ../nwoods_thesis.tex

\chapter{ZZ Phenomenology and Previous Results}

Four-lepton final states originate primarily from three physics processes: nonresonant diboson production, resonant Higgs boson production, and resonant single-{\PZ} production.
Multi-{\PZ} triboson production ({\WZZ} and {\ZZZ}) occurs at negligible rates~\cite{Lazopoulos:2007ix,Binoth:2008kt}.
Single-{\PZ} triboson production ({\WWZ})~\cite{Hankele:2007sb,Binoth:2008kt} and {\TTZ} production result in final states with four prompt leptons, but are considered background (see Section~\ref{sec:bkgPheno}).
The three signal processes can be distinguished by kinematics, but all involve the {\PZ} boson.

The {\PZ} was first observed in 1973 when the Gargamelle bubble chamber experiment at CERN recorded an elastic muon antineutrino-electron ($\Panm + \Pe^- \to \Panm + \Pe^-$) scattering event~\cite{Hasert:1973cr}.
Direct observation in leptonic decays came roughly a decade later, from the UA1 experiment, also at CERN~\cite{Arnison:1983mk}.
Clean $\Pe^+\Pe^-$ collisions at LEP and SLAC, where the center-of-mass energy could be adjusted to produce {\PZ} bosons copiously, allowed its properties---and a number of other parameters of the electroweak theory---to be measured with per-mille precision or better~\cite{ALEPH:2005ab}.
Of particular importance to this study, the {\PZ} mass is
\begin{equation}
  m_\PZ = 91.1876 \pm 0.0021\GeV,
\end{equation}
its full width is
\begin{equation}
  \Gamma_\PZ = 2.4952 \pm 0.0023\GeV,
\end{equation}
its width in leptonic decays is
\begin{equation}
  \Gamma_\PZ (\ell^+\ell^-) = 83.984 \pm 0.086\MeV,
\end{equation}
and it decays to a pair of charged leptons 3.3658\% of the time for each lepton flavor~\cite{Olive:2016xmw}.



\section[Nonresonant
         \texorpdfstring{$\mathrm{ZZ/Z}\gamma^\ast$}{ZZ/Zgamma*}
         Production and Decay]{Nonresonant $\mathbf{ZZ/Z}\gamma^\ast$ Production and Decay}

Leading-order {\ZZ} production is $\Pq\Paq$-initiated and proceeds through $t$-channel quark exchange, as shown in Fig.~\ref{fig:zzLO}.
At next-to-leading order (NLO\@; several representative diagrams are shown in Fig.~\ref{fig:zzNLO}), may have a gluon in the initial state and may have a quark or gluon in the final state which hadronizes and appears experimentally as a jet.
Next-to-next-to-leading order (NNLO) adds gluon-gluon fusion box diagrams (Fig.~\ref{fig:zzBox}), as well as $\Pq\Paq$-initiated production with two loops, one loop and a final state jet, and two jets.
The NNLO corrections are generally large, outside the scale uncertainties of the NLO calculation, because the gluon-fusion diagrams contribute only positively to the cross section because of their distinct initial state, and have a large amplitude---roughly 60\% of the total NNLO correction---due to the high effective gluon luminosity in multi-{\TeVns} proton collisions~\cite{Cascioli:2014yka}.
Because of the box diagrams' large contribution, ``$\text{NLO} + \Pg\Pg$'' simulations are often used, in which NLO $\Pq\Paq/\Pq\Pg/\Paq\Pg \to \ZZ$ and LO $\Pg\Pg \to \ZZ$ samples are summed even though they formally contribute at different orders in {\alphaS}.

\begin{figure}[htbp]
  \vspace{1em}
  \begin{center}
    \begin{fmffile}{zzLO}
      \begin{fmfgraph*}(0.2,0.2) % chktex 36
        \fmfstraight %chktex 1
        \fmfleft{i1,i2}
        \fmfright{o1,o2}
        \fmflabel{\Pq}{i1}
        \fmflabel{\Paq}{i2}
        \fmflabel{\PZ}{o1}
        \fmflabel{\PZ}{o2}
        \fmf{fermion}{i1,v1,v2,i2}
        \fmf{zigzag}{v1,o1}
        \fmf{zigzag}{v2,o2}
      \end{fmfgraph*}
    \end{fmffile}
    \hspace{4em}
      \begin{fmffile}{zzOL}
        \begin{fmfgraph*}(0.2,0.2) % chktex 36
          \fmfstraight %chktex 1
          \fmfleft{i1,i2}
          \fmfright{o1,o2}
          \fmflabel{\Pq}{i1}
          \fmflabel{\Paq}{i2}
          \fmflabel{\PZ}{o1}
          \fmflabel{\PZ}{o2}
          \fmf{fermion}{i1,v1,v2,i2}
          \fmf{phantom}{v1,o1}
          \fmf{phantom}{v2,o2}
          \fmffreeze % chktex 1
          \fmf{zigzag}{v1,o2}
          \fmf{zigzag}{v2,o1}
        \end{fmfgraph*}
      \end{fmffile}
    \vspace{1em}
    \caption[Leading order {\ZZ} production]{
      Leading order Feynman diagrams for {\ZZ} production in {\pp} collisions.
      }\label{fig:zzLO}
  \end{center}
\end{figure}

\begin{figure}[htbp]
  \vspace{1em}
  \begin{center}
    \begin{fmffile}{zzNLOg}
      \begin{fmfgraph*}(0.2,0.2) % chktex 36
        \fmfstraight %chktex 1
        \fmfleft{i1,d1,i2}
        \fmfright{o1,o2,o3}
        \fmflabel{\Pq}{i1}
        \fmflabel{\Paq}{i2}
        \fmflabel{\PZ}{o1}
        \fmflabel{\Pg}{o2}
        \fmflabel{\PZ}{o3}
        \fmf{fermion}{i1,v1,v2,v3,i2}
        \fmf{zigzag}{v1,o1}
        \fmf{zigzag}{v3,o3}
        \fmf{gluon}{v2,o2}
      \end{fmfgraph*}
    \end{fmffile}
    \hspace{4em}
    \begin{fmffile}{zzNLOLoop}
      \begin{fmfgraph*}(0.2,0.2) % chktex 36
        \fmfstraight %chktex 1
        \fmfleft{i1,d1,i2}
        \fmfright{o1,d2,o2}
        \fmflabel{\Pq}{i1}
        \fmflabel{\Paq}{i2}
        \fmflabel{\PZ}{o1}
        \fmflabel{\PZ}{o2}
        \fmf{fermion}{i1,v1,v2,v3,v4,i2}
        \fmf{gluon}{v1,v4}
        \fmf{zigzag}{v2,o1}
        \fmf{zigzag}{v3,o2}
      \end{fmfgraph*}
    \end{fmffile}
    \vspace{4em}

    \begin{fmffile}{zzNLOq}
      \begin{fmfgraph*}(0.2,0.2) % chktex 36
        \fmfstraight %chktex 1
        \fmfleft{i1,d1,i2}
        \fmfright{o1,o2,o3}
        \fmflabel{\Pq}{i1}
        \fmflabel{\Pg}{i2}
        \fmflabel{\PZ}{o1}
        \fmflabel{\PZ}{o2}
        \fmflabel{\Pq}{o3}
        \fmf{fermion}{i1,v1,v2,v3,o3}
        \fmf{zigzag}{v1,o1}
        \fmf{zigzag}{v2,o2}
        \fmf{gluon}{i2,v3}
      \end{fmfgraph*}
    \end{fmffile}
    \hspace{4em}
    \begin{fmffile}{zzNLOaq}
      \begin{fmfgraph*}(0.2,0.2) % chktex 36
        \fmfstraight %chktex 1
        \fmfleft{i1,d1,i2}
        \fmfright{o1,o2,o3}
        \fmflabel{\Paq}{i1}
        \fmflabel{\Pg}{i2}
        \fmflabel{\PZ}{o1}
        \fmflabel{\PZ}{o2}
        \fmflabel{\Paq}{o3}
        \fmf{fermion}{o3,v3,v2,v1,i1}
        \fmf{zigzag}{v1,o1}
        \fmf{zigzag}{v2,o2}
        \fmf{gluon}{i2,v3}
      \end{fmfgraph*}
    \end{fmffile}
    \vspace{1em}
    \caption[Next-to-leading order {\ZZ} production]{
      Three representative NLO Feynman diagrams for {\ZZ} production in {\pp} collisions.
      Clockwise from the top right, the NLO diagrams are examples of one-loop diagrams, real antiquark and quark emission, and real gluon emission.
      }\label{fig:zzNLO}
  \end{center}
\end{figure}

\begin{figure}[htbp]
  \vspace{1em}
  \begin{center}
    \begin{fmffile}{zzBox}
      \begin{fmfgraph*}(0.25,0.25) % chktex 36
        \fmfstraight %chktex 1
        \fmfleft{i1,i2}
        \fmfright{o1,o2}
        \fmflabel{\Pg}{i1}
        \fmflabel{\Pg}{i2}
        \fmflabel{\PZ}{o1}
        \fmflabel{\PZ}{o2}
        \fmf{gluon}{i1,v1}
        \fmf{gluon}{i2,v2}
        \fmf{zigzag}{v3,o2}
        \fmf{zigzag}{v4,o1}
        \fmf{fermion}{v3,v4,v1,v2}
        \fmf{fermion,label={\Pq}}{v2,v3}
      \end{fmfgraph*}
    \end{fmffile}
    \vspace{1em}
    \caption[Gluon-gluon fusion box diagram for {\ZZ} production]{
      A LO box diagram for {\ZZ} production through a quark loop in a gluon-gluon fusion event.
      This is formally an NNLO diagram for {\ZZ} production overall, but is often included in NLO calculations because it accounts for a large fraction of the NNLO correction, due to the high effective gluon luminosity in proton collisions at high $Q^2$.
      }\label{fig:zzBox}
  \end{center}
\end{figure}

Production of pairs of on-shell {\PZ} bosons\footnote{Events with two on-shell {\PZ} bosons are often called ``doubly resonant,'' but are a subset of ``nonresonant'' production in the sense that the {\ZZ} system is not produced by a resonance. Either term may be used to distinguish ``continuum'' production from ``singly resonant'' production from {\Zfourl}, $\PH \to \ZZ^\ast$, or a potential new particle which decays to {\ZZ}.} turns on sharply at the kinematic threshold $\sqrt{\hat{s}} = 2m_{\PZ} = 182.4\GeV$, and in proton-proton collisions at $\sqrt{s} = 13\TeV$, peaks around $m_{\ZZ} \approx 200\GeV$ before falling steeply at higher invariant masses.
Continuum production occurs below the kinematic threshold when one or both {\PZ} bosons are replaced by a $\PZ^\ast / \Pa^\ast$ admixture, typically in the form of a $\Pq\Paq \to \PZ$ event in which one of the incoming quarks emits a virtual photon as initial state radiation (ISR).
Events of interest in this analysis (see Sections~\ref{sec:zzSelection} and~\ref{sec:xSecCalc}) generally have one on-shell {\PZ}, and a $\PZ^\ast/\Pa^\ast$ at a lower mass.
Nonresonant $\PZ\Pa^\ast$ production is generally flat as a function of invariant mass between roughly {100\GeV} and the doubly resonant threshold.


\subsection{Vector Boson Scattering}

Vector boson scattering proceeds at hadron colliders through the diagrams shown in Fig~\ref{fig:vbs}, resulting in a {\ZZjj} final state.
This fully electroweak (EWK) production must be distinguished from the background of QCD-initiated $\ZZ + \text{jets}$ events.
The hallmark of the EWK process is a pair of high energy, high rapidity jets from the quarks, which retain a high boost along the $z$-axis even after {\PWpm} or {\PZ} emission and are thus deflected through a small angle in the lab frame.
At the same time, the {\ZZ} system is produced with high invariant mass and low boost compared to QCD-initiated {\ZZjj} events, in which the {\ZZ} system recoils against the jets~\cite{Zeppenfeld:54.6680}.
Because the hard scattering interaction involves no color exchange or reconnection, VBS events are much less likely to have less energetic jets between the two high-energy quark jets.
Useful variables to discriminate between EWK and QCD production therefore include the angle between the jets and their energy, their invariant mass, the {\ZZ} invariant mass and rapidity, and the number of central jets.

\begin{figure}[htbp]
  \vspace{1em}
  \begin{center}
    \begin{fmffile}{vbsTGC}
      \begin{fmfgraph*}(0.3,0.25) % chktex 36
        \fmfstraight %chktex 1
        \fmfleft{i1,d1,d2,i2}
        \fmfright{o1,o2,o3,o4}
        \fmflabel{\Pq}{i1}
        \fmflabel{\Pq}{i2}
        \fmflabel{$\Pq'$}{o1}
        \fmflabel{$\Pq'$}{o4}
        \fmflabel{\PZ}{o2}
        \fmflabel{\PZ}{o3}
        \fmf{fermion}{i1,v1,o1}
        \fmf{zigzag}{v1,v2}
        \fmf{zigzag,label={\PWpm},label.side=left}{v2,v3}
        \fmf{zigzag}{v3,v4}
        \fmf{fermion}{i2,v4,o4}
        \fmf{phantom}{d1,v1}
        \fmf{phantom}{d2,v4}
        \fmffreeze % chktex 1
        \fmf{zigzag}{v2,o2}
        \fmf{zigzag}{v3,o3}
      \end{fmfgraph*}
    \end{fmffile}
    \hspace{4em}
    \begin{fmffile}{vbsQGC}
      \begin{fmfgraph*}(0.3,0.25) % chktex 36
        \fmfstraight %chktex 1
        \fmfleft{i1,d1,d2,i2}
        \fmfright{o1,o2,o3,o4}
        \fmflabel{\Pq}{i1}
        \fmflabel{\Pq}{i2}
        \fmflabel{$\Pq'$}{o1}
        \fmflabel{$\Pq'$}{o4}
        \fmflabel{\PZ}{o2}
        \fmflabel{\PZ}{o3}
        \fmf{fermion}{i1,v1,o1}
        \fmf{phantom}{d1,v1}
        \fmf{phantom}{d2,v3}
        \fmf{phantom}{v1,v4,v5,v3}
        \fmf{fermion}{i2,v3,o4}
        \fmffreeze % chktex 1
        \fmf{zigzag,label={\PWpm}}{v1,v2}
        \fmf{zigzag,label={\PWmp}}{v2,v3}
        \fmf{zigzag}{o2,v2,o3}
      \end{fmfgraph*}
    \end{fmffile}
    \vspace{4em}

    \begin{fmffile}{vbsHt}
      \begin{fmfgraph*}(0.3,0.25) % chktex 36
        \fmfstraight %chktex 1
        \fmfleft{i1,d1,d2,i2}
        \fmfright{o1,o2,o3,o4}
        \fmflabel{\Pq}{i1}
        \fmflabel{\Pq}{i2}
        \fmflabel{\Pq}{o1}
        \fmflabel{\Pq}{o4}
        \fmflabel{\PZ}{o2}
        \fmflabel{\PZ}{o3}
        \fmf{fermion}{i1,v1,o1}
        \fmf{fermion}{i2,v4,o4}
        \fmf{phantom}{v1,v5,v6,v4}
        \fmf{phantom}{d1,v1}
        \fmf{phantom}{d2,v4}
        \fmffreeze % chktex 1
        \fmf{zigzag}{v1,v2,o2}
        \fmf{dashes,label={\PH},label.side=left}{v2,v3}
        \fmf{zigzag}{v4,v3,o3}
      \end{fmfgraph*}
    \end{fmffile}
    % \hspace{4em}
    % \begin{fmffile}{vbsHs}
    %   \begin{fmfgraph*}(0.3,0.25) % chktex 36
    %     %\fmfstraight %chktex 1
    %     \fmfleft{i1,d1,d2,i2}
    %     \fmfright{o1,o2,o3,o4}
    %     \fmflabel{\Pq}{i1}
    %     \fmflabel{\Pq}{i2}
    %     \fmflabel{\Pq}{o1}
    %     \fmflabel{\Pq}{o4}
    %     \fmflabel{\PZ}{o2}
    %     \fmflabel{\PZ}{o3}
    %     \fmf{fermion}{i1,v1,o1}
    %     \fmf{zigzag,label=$\PWpm,,\PZ$,label.side=left}{v1,v2}
    %     \fmf{zigzag,label=$\PWmp,,\PZ$,label.side=left}{v2,v4}
    %     \fmf{dashes,label={\PH}}{v2,v3}
    %     \fmf{zigzag}{o2,v3,o3}
    %     \fmf{fermion}{i2,v4,o4}
    %   \end{fmfgraph*}
    % \end{fmffile}
    \vspace{1em}
    \caption[Vector boson scattering diagrams]{
      The primary {\ZZ} VBS diagrams at hadron colliders.
      Diagrams also exist with antiquarks and with permutation and crossing of the final state particles.
      The interaction is only unitary to arbitrarily high energy when all diagrams are considered.
      }\label{fig:vbs}
  \end{center}
\end{figure}


\subsection{Prior Measurements}
The literature before I got here


\section[Resonant
         \texorpdfstring{$\mathrm{ZZ}^\ast$/$\mathrm{Z\gamma}^\ast$}
         {ZZ*/Zgamma*}
         Production]{Resonant $\mathbf{ZZ}^\ast$/$\mathbf{Z\gamma}^\ast$ Production}
This resonates with me

\subsection{Z Boson Decays to Four Leptons}
Prior measurements of this probably don't need their own subsubsection


\subsection{Higgs Boson Production}\label{sec:Hproduction}

\subsubsection{Prior Measurements}\label{sec:Hresults}
Discovery!



\section{Anomalous Gauge Couplings}
Triple and quartic

\subsection{Previous Limits}
Pro tip: they aren't there



\section{Background Processes}\label{sec:bkgPheno}
Basically, Z+jets and ttbar
