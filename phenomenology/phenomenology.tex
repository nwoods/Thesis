%!TEX root = ../nwoods_thesis.tex

\chapter{ZZ Phenomenology and Previous Results}

Four-lepton final states originate primarily from three physics processes: nonresonant diboson production, resonant Higgs boson production, and resonant single-{\PZ} production.
Multi-{\PZ} triboson production ({\WZZ} and {\ZZZ}) occurs at negligible rates~\cite{Lazopoulos:2007ix,Binoth:2008kt}.
Single-{\PZ} triboson production ({\WWZ})~\cite{Hankele:2007sb,Binoth:2008kt} and {\TTZ} production result in final states with four prompt leptons, but are considered background (see Section~\ref{sec:bkgPheno}).
The three signal processes can be distinguished by kinematics, but all involve the {\PZ} boson.

The {\PZ} was first observed in 1973 when the Gargamelle bubble chamber experiment at CERN recorded an elastic muon antineutrino-electron ($\Panm + \Pe^- \to \Panm + \Pe^-$) scattering event~\cite{Hasert:1973cr}.
Direct observation in leptonic decays came roughly a decade later, from the UA1 experiment, also at CERN~\cite{Arnison:1983mk}.
Clean $\Pe^+\Pe^-$ collisions at LEP and SLAC, where the center-of-mass energy could be adjusted to produce {\PZ} bosons copiously, allowed its properties---and a number of other parameters of the electroweak theory---to be measured with per-mille precision or better~\cite{ALEPH:2005ab}.
Of particular importance to this study, the {\PZ} mass is
\begin{equation}
  m_\PZ = 91.1876 \pm 0.0021\GeV,
\end{equation}
its full width is
\begin{equation}
  \Gamma_\PZ = 2.4952 \pm 0.0023\GeV,
\end{equation}
its width in leptonic decays is
\begin{equation}
  \Gamma_\PZ (\ell^+\ell^-) = 83.984 \pm 0.086\MeV,
\end{equation}
and it decays to a pair of charged leptons 3.3658\% of the time for each lepton flavor~\cite{Olive:2016xmw}.



\section[Nonresonant
         \texorpdfstring{$\mathrm{ZZ/Z}\gamma^\ast$}{ZZ/Zgamma*}
         Production and Decay]{Nonresonant $\mathbf{ZZ/Z}\gamma^\ast$ Production and Decay}

Leading-order {\ZZ} production is $\Pq\Paq$-initiated and proceeds through $t$-channel quark exchange, as shown in Fig.~\ref{fig:zzLONLO}.
At next-to-leading order (representative diagrams also in Fig.~\ref{fig:zzLONLO}), may have a gluon in the initial state and may have a quark or gluon in the final state which hadronizes and appears as a jet.

\begin{figure}[htbp]
  \vspace{1em}
  \begin{center}
    \begin{fmffile}{zzLO}
      \begin{fmfgraph*}(0.2,0.2) % chktex 36
        \fmfstraight %chktex 1
        \fmfleft{i1,i2}
        \fmfright{o1,o2}
        \fmflabel{\Pq}{i1}
        \fmflabel{\Paq}{i2}
        \fmflabel{\PZ}{o1}
        \fmflabel{\PZ}{o2}
        \fmf{fermion}{i1,v1,v2,i2}
        \fmf{zigzag}{v1,o1}
        \fmf{zigzag}{v2,o2}
      \end{fmfgraph*}
    \end{fmffile}
    \hspace{4em}
    \begin{fmffile}{zzNLOLoop}
      \begin{fmfgraph*}(0.2,0.2) % chktex 36
        \fmfstraight %chktex 1
        \fmfleft{i1,d1,i2}
        \fmfright{o1,d2,o2}
        \fmflabel{\Pq}{i1}
        \fmflabel{\Paq}{i2}
        \fmflabel{\PZ}{o1}
        \fmflabel{\PZ}{o2}
        \fmf{fermion}{i1,v1,v2,v3,v4,i2}
        \fmf{gluon}{v1,v4}
        \fmf{zigzag}{v2,o1}
        \fmf{zigzag}{v3,o2}
      \end{fmfgraph*}
    \end{fmffile}
    \vspace{4em}

    \begin{fmffile}{zzNLOg}
      \begin{fmfgraph*}(0.2,0.2) % chktex 36
        \fmfstraight %chktex 1
        \fmfleft{i1,d1,i2}
        \fmfright{o1,o2,o3}
        \fmflabel{\Pq}{i1}
        \fmflabel{\Paq}{i2}
        \fmflabel{\PZ}{o1}
        \fmflabel{\Pg}{o2}
        \fmflabel{\PZ}{o3}
        \fmf{fermion}{i1,v1,v2,v3,i2}
        \fmf{zigzag}{v1,o1}
        \fmf{zigzag}{v3,o3}
        \fmf{gluon}{v2,o2}
      \end{fmfgraph*}
    \end{fmffile}
    \hspace{4em}
    \begin{fmffile}{zzNLOq}
      \begin{fmfgraph*}(0.2,0.2) % chktex 36
        \fmfstraight %chktex 1
        \fmfleft{i1,d1,i2}
        \fmfright{o1,o2,o3}
        \fmflabel{\Pq}{i1}
        \fmflabel{\Pg}{i2}
        \fmflabel{\PZ}{o1}
        \fmflabel{\PZ}{o2}
        \fmflabel{\Pq}{o3}
        \fmf{fermion}{i1,v1,v2,v3,o3}
        \fmf{zigzag}{v1,o1}
        \fmf{zigzag}{v2,o2}
        \fmf{gluon}{i2,v3}
      \end{fmfgraph*}
    \end{fmffile}
    \vspace{1em}
    \caption[Leading order and next-to-leading order {\ZZ} production]{
      Leading order (top left) and three representative next-to-leading order Feynman diagrams for {\ZZ} production in {\pp} collisions.
      Clockwise from the top right, the NLO diagrams are representative of one-loop diagrams, real quark emission, and real gluon emission.
      }\label{fig:zzLONLO}
  \end{center}
\end{figure}


\subsection[Nonresonant
            \texorpdfstring{$\mathrm{Z}\gamma^\ast$}{Zgamma*}
            Production]{Nonresonant $\mathbf{Z}\gamma^\ast$ Production}
Virtual particles. Spoooooooooky!

\subsection{Vector Boson Scattering}
Them jets tho

\subsection{Prior Measurements}
The literature before I got here


\section[Resonant
         \texorpdfstring{$\mathrm{ZZ}^\ast$/$\mathrm{Z\gamma}^\ast$}
         {ZZ*/Zgamma*}
         Production]{Resonant $\mathbf{ZZ}^\ast$/$\mathbf{Z\gamma}^\ast$ Production}
This resonates with me

\subsection{Z Boson Decays to Four Leptons}
Prior measurements of this probably don't need their own subsubsection


\subsection{Higgs Boson Production}\label{sec:Hproduction}

\subsubsection{Prior Measurements}\label{sec:Hresults}
Discovery!



\section{Anomalous Gauge Couplings}
Triple and quartic

\subsection{Previous Limits}
Pro tip: they aren't there



\section{Background Processes}\label{sec:bkgPheno}
Basically, Z+jets and ttbar
